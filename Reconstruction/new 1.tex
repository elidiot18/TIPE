\documentclass{report}
\usepackage[T1]{fontenc}
\usepackage[frenchb]{babel}
\usepackage{xcolor}
\usepackage[a4paper]{geometry}
\geometry{top=1.5cm, bottom=1.5cm, left=2cm, right=2cm}
\usepackage{amsmath}
\usepackage{amssymb}
\usepackage{amsthm}
\usepackage{enumerate}
\usepackage{hyperref}
\let\endtitlepage\relax

\newcommand{\R}{\mathbb{R}}
\newcommand{\Vor}{\mathrm{Vor}}
\newcommand{\Del}{\mathrm{Del}}
\newcommand{\Sql}{\mathrm{Sql}}
\newcommand{\eps}{\varepsilon}
\newcommand{\lfs}{\mathrm{lfs}}

\newtheorem{theorem}{Théorème}
\newtheorem{definition}{Définition}
\newtheorem{lemma}{Lemme}
\newtheorem{prop}{Proposition}
\theoremstyle{remark}
\newtheorem*{rmq}{Remarque}

\title{\textbf{Reconstruction de formes multi-échelles en dimensions deux et trois grâce à la strucure de complexes de témoins}}
\author{CHALUMEAU Adam,	ELIS Pierre}
\date{\today}

\begin{document}
\maketitle

\appendix
\chapter{Définitions, lemmes et théorèmes}
\paragraph{Résumé}
\begin{quote}
{\small Dans cette première annexe se trouvent toutes les définitions additionnelles à celles du programme dont nous avons eu besoin pour traiter pleinement notre sujet de TIPE. Quelques propriétés viennent jusitifer certaines définitions. En outre cette annexe contient les lemmes utiles à la démonstration des deux théorèmes principaux de notre étude, ainsi que ceux-ci et leur preuves respectives.}
\end{quote}

\section{Définitions basiques}
\begin{definition}
	Un \textnormal{simplexe} de $\R^d$ est l'enveloppe convexe de $d+1$ points affinement indépendants qui sont appelés ses sommets.
\end{definition}

\begin{definition}
	Une \textnormal{$k$-face} d'un simplexe de $\R^d$ est l'enveloppe convexe de $k+1$ sommets de $P$. Si cette enveloppe convexe n'est pas $P$ tout entier on parle de \textnormal{face propre}
\end{definition}

\begin{rmq} 
	Une $k$-face d'un simplexe de $\R^d$ est un simplexe de $\R^k$
\end{rmq}

\begin{definition}
	Un \textnormal{complexe simplicial} $C$ dans $\R^d$ est un ensemble de simplexes de $\R^d$ qui vérifie
	\begin{enumerate}[(i)]
		\item toute face d'un simplexe de $C$ est elle-même un simplexe de $C$
		\item l'intersection de deux simplexes de $C$ est soit vide soit une face commune à ces deux simplexes
	\end{enumerate}
\end{definition}

\begin{definition}
	Une \textnormal{triangulation} d'un nuage de points $P$ de $\R^d$ est un complexe simplicial $T$ tel que $\bigcup\limits_{t\in T}t = P$ et tel que tout simplexe de $t$ a ses sommets dans $P$
\end{definition}

\section{La triangulation de Delaunay et le diagramme de Voronoi}
\begin{definition}
	Soit $E$ un ensemble. On appelle \textnormal{nuage de points} de $E$ tout sous ensemble fini de $E$
\end{definition}

\begin{definition}
	Soit $P$ un nuage de points de $\R^d$ et $\sigma = (p_1, p_2, \cdots, p_n)$ un simplexe à sommets dans $P$. On dit que $\sigma$ est un \textnormal{simplexe de Delaunay} s'il existe une boule circonscrite à $\sigma$, dite \textnormal{boule de Delaunay}, dont l'intérieur est vide de tous points de $P$, i.e.
	$$\exists c\in \R^d, \, \exists r > 0, \, p_1, p_2, \cdots, p_n \in \mathcal S(c, r) \text{ et }\mathcal B(c, r)\cap P = \emptyset$$
\end{definition}

\begin{prop}
	Soit $P$ un nuage de points de $\R^d$. L'ensemble des simplexes de Delaunay $\Del(P)$ du nuage de points $P$ est une triangulation de $P$.
\end{prop}

\begin{proof}
	Démonstration à réaliser. \#espace des sphères
\end{proof}

\begin{definition}
	Soit $P$ un nuage de points de $\R^d$. La triangulation $\Del(P)$  de $P$ est appelée \textnormal{triangulation de Delaunay} $P$
\end{definition}

\begin{definition}
	Soit $P$ un nuage de points de $\R^d$. Soit $p\in P$. On appelle \textnormal{cellule de Voronoi} de $p$ dans $P$ l'ensemble $\Vor_p(P) = \{q\in \R^d\,|\,\forall x\in P, d(x, p) \leq d(x, q)\}$ c'est à dire l'ensemble des points de $\R^d$ qui sont au moins aussi proches de $p$ que des autres points de $P$
\end{definition}

\begin{definition}
	Soit $P$ un nuage de points de $\R^d$. On appelle \textnormal{diagramme de Voronoi} de $P$ l'ensemble $\Vor(P) = \{\Vor_p(P)\,|\,p\in P\}$
\end{definition}

\begin{definition}
	Soit $P$ un nuage de points de $\R^d$, on appelle \textnormal{arête de Voronoi} de $P$ toute intersection non vide d'élements de $\Vor(P)$
\end{definition}

\begin{prop}
	La triangulation de Delaunay d'un nuage de points et son diagramme de Voronoï sont duaux l'un de l'autre.
\end{prop}

\begin{proof}
	Preuve à insérer.
\end{proof}

\section{Reconstruction}
\begin{definition}
	 Soient $S \subset \R^d$ et $\eps>0$. Soit $W\subset \R^d$ un ensemble fini.
	\begin{enumerate}
	\item On dit que $W$ est un échantillon \textnormal{$\eps$-bruité} de $S$ si $\forall w \in W,\, d(w, S) \leq \eps$
	\item On dit que $W$ est un \textnormal{$\eps$-échantillon} de $S$ si : $\forall s \in S,\, d(s, W) \leq \eps$
	\item On dit que $W$ est un échantillon \textnormal{$\eps$-creux} de $S$ si : $\forall (w, s) \in W\times S,\, d(w, s) \leq \eps$
	\end{enumerate}
\end{definition}

\begin{rmq}
	Jusqu'à présent nous avons donné des définitions générales dans $\R^d$. Néanmoins le bagage mathématique nécessaire aux démonstrations de tous les théorèmes en dimension quelconque dépassent largement les prétentions de notre TIPE et nous envisagerons désormais des formes géométriques dans le plan $\R^2$. Nous nous placerons toutefois par moments en dimension supérieure.
\end{rmq}

\begin{definition}
	On appelle courbe fermée de $\R^2$ tout ensemble $S\subset\R^2$ homéomorphe à la sphère unité.
\end{definition}

\begin{definition}
	Soit $S$ un sous-ensemble de $\R^d$, soient $x\in\R^d$ et $s_0\in S$. On dit $s_0$ est un \textnormal{plus proche voisin} de $x$ dans $S$ si : $\forall s\in S, d(x, s_0) \leq d(x, s)$
\end{definition}
	 
\begin{definition}
	Soit $S$ un sous-ensemble de $\R^d$. On appelle \textnormal{axe médian} de $S$ l'ensemble $M_S$ des points de $x$ de $\R^d$ tels que $x$ a au moins deux plus proches voisins distincts dans $S$
\end{definition}

\begin{definition}
	Soit $S$ un sous-ensemble de $\R^d$. On appelle \textnormal{quelette} de $S$ l'ensemble $\Sql_S$ adhérence de l'ensemble des centres de boules ouvertes maximales incluses dans $\R^d\setminus S$, c'est à dire, l'ensemble des centres des boules $B\subset\R^d\setminus S$ telles que pour toute autre boule $B^\prime\subset\R^d\setminus S$, si $B\subset B^\prime$ alors $B=B^\prime$
\end{definition}

\begin{prop}
	On a l'inclusion suivante : $\Sql_S\subset M_S$. L'égalité est en fait vérifiée mais la démonstration est trop technique pour cet exposé, de plus nous n'utiliserons que la première inclusion dans nos démonstrations.
\end{prop}

\begin{proof}
	Preuve à insérer.
\end{proof}

\begin{definition}
	Soit $S$ un sous-ensemble de $\R^d$. On appelle \textnormal{local feature size} la fonction qui à $s\in S$ associe la distance de $s$ à $M_S$ et on la note $\mathrm{lfs}_S$.
\end{definition}

\begin{prop}
	Soit $S$ un sous-ensemble de $\R^d$. $\lfs_S$ est continue (en fait 1-lipschitzienne).
\end{prop}

\begin{proof}
	Preuve à insérer, ou pas.
\end{proof}

\begin{definition}
	Si $S$ est un compact de $\R^d$, on appelle \textnormal{reach} et on note $\rho_S = \min_S\mathrm{lfs}_S$. On dit qu'une forme gémométrique est à \textnormal{reach positif} si son reach est strictement positif.
\end{definition}

\begin{rmq}
	\`A titre informatif, une forme géométrique à reach positif est au moins une fois différentiable et donc relativement lisse. Cette propriété ne nous sert pas dans cet exposé et sa démonstration utilise des outils de géométrie différentielle que nous avons décidé de ne pas étudier.
\end{rmq}

\begin{definition}
	On appelle \textnormal{arc} toute fonction continue de $[0,1]$ dans $\R^d$
\end{definition}

\begin{definition}
	Soit $S$ une courbe fermée de $\R^2$ à reach positif. Soient $x\in S$ et $r\leq\lfs_S(x)$. On appelle \textnormal{disques tangeants à $S$ en $x$} les deux disques... \textbf{DEFINITION A TERMINER + PREUVE}
\end{definition}

\begin{lemma}
	Soit $S$ une courbe fermée de $\R^2$ à reach positif. Soient $x\in S$ et $r\leq\lfs_S(x)$. Soient $D_1$ et $D_2$ les deux disques tangeants à $S$ en $x$. Alors $D_1\cap S\neq\emptyset$ et $D_2\cap S\neq\emptyset$.
\end{lemma}

\begin{proof}
	Preuve à insérer.
\end{proof}

\begin{lemma}
	Soit $S$ une courbe fermée de $\R^2$ à reach positif. Soit $D$ un disque fermé tel que $D\cap S\neq\emptyset$ et $D\cap S$ n'est pas un arc. Alors $D\cap M_S\neq\emptyset$.
\end{lemma}

\begin{definition}[Voronoi et Delaunay restreints]
	Soit $S$ une courbe fermée de $\R^d$, soit $P$ un nuage de points de $S$. 
	\begin{enumerate}
		\item On appelle \textnormal{Voronoi restreint à $S$ de $P$} l'ensemble $\Vor_S(P)$ des cellules de Voronoi de $P$ qui intersectent $S$ c'est à dire $\Vor_S(P) = \{f\in\Vor(P)\,|\,f\cap S\neq\emptyset\}$. 
		\item On appelle \textnormal{Delaunay restreint à $S$ de $P$} le sous-complexe $\Del_S(P)$ de $\Del(P)$ dual de $\Vor_S(P)$
	\end{enumerate}
\end{definition}

\begin{definition}
	Soit $S = s([0,1])$ une courbe fermée continue de $R^2$ et soient $p, \, q\in S$. On appelle \textnormal{arc de $S$ reliant $p$ à $q$} l'ensemble $s\left(\left[s^{-1}(p), s^{-1}(q)\right]\right)$.
\end{definition}

\begin{definition}
	Soit $S = s([0,1])$ une courbe fermée continue de $R^2$, soit $W$ un nuage de points de $S$ et soient $p, \, q\in W$. On dit que $p$ et $q$ sont \textnormal{consécutifs le long de $S$} si l'intérieur de l'arc de $S$ reliant $p$ à $q$ est vide de points de $W$.
\end{definition}

\begin{lemma}
	Soit $S$ une courbe fermée de $R^2$ à reach positif, soient $\eps>0$ et $W$ un $\eps-échantillon$ de $S$ avec $\eps<\rho_S$. Soit $e = [p, q]$ une arête de $\Del_S(P)$. Alors $p$ et $q$ sont consécutifs le long de $S$ et si on note $c$ l'arc de $S$ reliant $p$ à $q$ et $v$ l'arête de Voronoï duale de $[p, q]$ alors on a $S\cap v\subset c$.
\end{lemma}

\begin{proof}
	Preuve à insérer.
\end{proof}

\begin{theorem}
	Soit $S$ une courbe fermée de $R^2$ à reach positif, soient $\eps>0$ et $W$ un $\eps-échantillon$ de $S$ avec $\eps<\rho_S$. Alors, $\Del_S(W)$ est homéomorphe à $S$.
\end{theorem}

\begin{proof}
	Preuve à insérer.
\end{proof}


\begin{proof}
	Preuve à insérer.
\end{proof}

\begin{definition}[Complexe de témoins]
	Soient $W$ et $L$ deux nuages de points de $\R^d$.
	\begin{itemize}
	\item[$\bullet$] Soit $w\in W$ et soit $\sigma = (p_0, p_2, \cdots, p_n)$ un simplexe à sommets dans $L$. On dit que $w$ est un \textnormal{témoin} de $\sigma$ ou que $w$ \textnormal{témoigne} $\sigma$ si les sommets de $\sigma$ sont les $n+1$ plus proches voisins de $w$ dans $L$, i.e. : $\forall k\in[\![1,n]\!],\, \forall p\in L,\, d(w, p_k)\leq d(w, p)$
	\item[$\bullet$] On appelle \textnormal{complexe de témoins} et on note $C^W(L)$ le complexe simplicial de plus grand cardinal tel que tous ses simplexes aient un témoin dans $L$
	\end{itemize}
\end{definition}

\begin{rmq}
	Les points de $L$ ont vocation à être appelés landmarks oupoints repère en version francisée.
\end{rmq}


\end{document}
